\documentclass[12pt, a4paper]{report}
\usepackage[utf8]{inputenc}
\usepackage[english, russian]{babel}

\usepackage{graphicx}
\usepackage{listings}
\usepackage{color}

\usepackage{amsmath}
\usepackage{pgfplots}
\usepackage{url}
\usepackage{flowchart}
\usepackage{tikz}
\DeclareGraphicsExtensions{.pdf,.png,.jpg,.svg}
\usetikzlibrary{shapes, arrows}

\usepackage{pgfplotstable}

\renewcommand\contentsname{Содержание}

\usepackage{geometry}
\geometry{left=3cm}
\geometry{right=1cm}
\geometry{top=2cm}
\geometry{bottom=2cm}

\lstset{ %
language=C++,                 % выбор языка для подсветки (здесь это С)
basicstyle=\small\sffamily, % размер и начертание шрифта для подсветки кода
numbers=left,               % где поставить нумерацию строк (слева\справа)
numberstyle=\tiny,           % размер шрифта для номеров строк
stepnumber=1,                   % размер шага между двумя номерами строк
numbersep=-5pt,                % как далеко отстоят номера строк от         подсвечиваемого кода
backgroundcolor=\color{white}, % цвет фона подсветки - используем         \usepackage{color}
showspaces=false,            % показывать или нет пробелы специальными     отступами
showstringspaces=false,      % показывать или нет пробелы в строках
showtabs=false,             % показывать или нет табуляцию в строках
frame=single,              % рисовать рамку вокруг кода
tabsize=2,                 % размер табуляции по умолчанию равен 2 пробелам
captionpos=t,              % позиция заголовка вверху [t] или внизу [b] 
breaklines=true,           % автоматически переносить строки (да\нет)
breakatwhitespace=false, % переносить строки только если есть пробел
escapeinside={\%*}{*)},   % если нужно добавить комментарии в коде
keywordstyle=\color{blue}\ttfamily,
stringstyle=\color{red}\ttfamily,
commentstyle=\color{green}\ttfamily,
morecomment=[l][\color{magenta}]{\#},
columns=fullflexible }

\usepackage{titlesec}
\titleformat{\chapter}[hang]{\LARGE\bfseries}{\thechapter{.} }{0pt}{\LARGE\bfseries}
\titleformat*{\section}{\Large\bfseries}
\titleformat*{\subsection}{\large\bfseries}

\begin{document}

    \begin{titlepage}

        \begin{center}
            \Large
            {\sl Государственное образовательное учреждение высшего профессионального образования\\
            {\bf«Московский государственный технический университет имени Н.Э. Баумана»\\
				(МГТУ им. Н.Э. Баумана)}}
            \vspace{3cm}

			{\scshape\LARGE Лабораторная работа №... \par}
			\vspace{0.5cm}	
			{\scshape\LARGE по курсу «Анализ алгоритмов» \par}
			\vspace{1.5cm}
			{\huge\bfseries Тема лабораторной работы \par}
			\vspace{2cm}
			\Large Выполнил: Тимонин А.С., гр. ИУ7-52Б\\
			\vspace{0.5cm}
			{\Large Преподаватели: Волкова Л.Л., Строганов Ю.В.}
		
			\vfill
			\Large \textit {2019 г.}
            
        \end{center}

    \end{titlepage}
	
	\tableofcontents

	\chapter*{Введение}
	\addcontentsline{toc}{chapter}{Введение}
	

    \chapter{Аналитический раздел}
   	\vspace{-0.5cm}бла-бла-бла
	\section{Описание алгоритмов}
	бла-бла-бла

	\section{Алгоритм1}
	
	\section{Алгоритм2}
	бла-бла-бла
	

	\chapter{Конструкторский раздел}
	
	\vspace{-0.5cm} bla-bla-bla
	
	\section{Разработка алгоритмов}
	тут схема алгоритмов

	\newpage

	\section{Расчет сложности}
	бла-бла-бла
	
	\section{Вывод}
	бла-бла-бла
	
	\begin{itemize}
		\item бла-бла-бла
		\item бла-бла-бла
		\item бла-бла-бла
	\end{itemize}
	
	\chapter{Технологический раздел}
	\vspace{-0.5cm}бла-бла-бла
	
	\section{Требования к программному обеспечению}
	бла-бла-бла
	\section{Средства реализации}
	Для выполнения поставленной задачи был использован язык программирования С++. Среда для разработки XCode. Для измерения процессорного времени была взята функция rdtsc из библиотеки ctime.
	
	\vspace{0.2cm}Данный язык обусловлен тем, что функции замеры времени могут считывать не только абсолютное время, но и процессорное.
	
	\vspace{0.2cm}Версия компилятора C++: GNU++14 [-std=gnu++14]
	
	
	
	\section{Листинг кода}
	бла-бла-бла

	\begin{lstlisting}[label=code-opt,caption=бла-бла-бла]
	void name(int a)
	{
		return;
	}
	\end{lstlisting}

	\newpage

	\section{Вывод}
	бла-бла-бла

			
	\chapter{Экспериментальный раздел}
	
	\vspace{-0.5cm}бла-бла-бла
	
	\section{Сравнительный анализ}
	Замеры времени выполнялись на тестах...... Все замеры проводились на процессоре 1,4 GHz Intel Core i5 с памятью 8 ГБ 2133 MHz LPDDR3.
	
	\begin{table}[ht!]
		\caption{бла-бла-бла}
		\label{unit-tests}
		\begin{center}
			\begin{tabular}{|c|c|c|}
				\hline
				\bf{кек1} & \bf{кек2} & \bf{кек3}\\\hline
				
				$1$ & $2$ & $3$\\\hline
				
				$1$ & $2$ & $3$\\\hline
			\end{tabular}
		\end{center}
	\end{table}
	
	
	\section{Вывод}
	бла-бла-бла

	\newpage
	
 	бла-бла-бла

	\chapter*{Заключение}
	\addcontentsline{toc}{chapter}{Заключение}
	бла-бла-бла
	
	\newpage
	
	\begin{thebibliography}{}
	\bibitem ...
	\end{thebibliography}
	\addcontentsline{toc}{chapter}{Литература}

\end{document}